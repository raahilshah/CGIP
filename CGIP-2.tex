\documentclass[11pt]{article}

\usepackage[a4paper]{geometry}
\usepackage{amsmath, amsthm, amssymb, amsthm, graphicx}
\renewcommand\labelenumi{\alph{enumi}.}
\renewcommand\labelenumii{\roman{enumii}.}
\renewcommand\labelitemi{--}


\title{1B - Computer Graphics and Image Processing\\
Supervision 2}
\author{Raahil Shah\\
\texttt{rds46@cam.ac.uk}}

\begin{document}

\maketitle

\section{Matrices}
Give as many reasons as possible why we use matrices to represent transformations. Explain why we use homogeneous co-ordinates

\section{Bezier cubics}
Derive the conditions necessary for two Bezier curves to join with 
\begin{itemize}
	\item Just C0-continuity
	\item C1-continuity
	\item C2-continuity. 
\end{itemize}
What would be difficult about getting three Bezier curves to join in sequence with C2-continuity at the two joins?

\section{Curves}
Implement the Midpoint circle drawing algorithm for integer radius and integer center location. 
Hints: 
\begin{itemize}
	\item Implement the algorithm for one octant, centered on the origin, then draw the other octants by manipulating the co-ordinates of the drawn point: (x,y), (-x,y), (y,x), (y,-x), etc.
	\item Implement the algorithm for a circle centered on the origin and then simply offset the pixel that is drawn by the coordinates of the actual center.
\end{itemize}

Write the Bézier adaptive algorithm, using your Midpoint line drawing to draw the lines. To test this on a pixel grid of size HxW, try the Bézier curve specified by the four points: (1,1), (2H,1), (-H,W-1), (H-1,W-1). Draw the same Bézier curve at five different tolerances: 33, 10, 3.3, 1, 0.33 pixels.
Draw something interesting!



\end{document}